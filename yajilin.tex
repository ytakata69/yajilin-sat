\documentclass[a4j]{jarticle}
\usepackage{amsmath}

\parindent=0pt
\setcounter{secnumdepth}{2}

\newcommand{\N}[1]{\overline{#1\vphantom{l}}}
\newcommand{\←}{{\leftarrow}}
\newcommand{\→}{{\rightarrow}}
\newcommand{\↑}{{\uparrow}}
\newcommand{\↓}{{\downarrow}}

\begin{document}

\section*{ヤジリンをSATソルバで解く方法}
\begin{flushright}
2018.8.29 y-takata
\end{flushright}

ヤジリンの遊び方,ルール,解き方\\
https://www.nikoli.co.jp/ja/puzzles/yajilin/

\section{準備}
盤面の幅を$W$,高さを$H$とする.
第$i$列第$j$行のマスをマス$(i,j)$と呼ぶ ($1\le i\le W$, $1\le j\le H$).

$X=\{\←,\→,\↑,\↓\}$とする.
$X$の各要素はマスからマスへの関数.例えば$\→(i,j)=(i+1,j)$.
%$X(i,j)=\{(i-1,j),(i,j-1),(i+1,j),(i,j+1)\}$.

\section{SAT符号化}
\subsection{命題変数}
\begin{itemize}
\item $b_{i,j}$: マス$(i,j)$は黒.
\item $d_{i,j}$: マス$(i,j)$は数字のマス.
\item $l_{i,j,a}$: マス$(i,j)$がマス$a(i,j)$と線でつながっている ($a\in X$).
\item $c_{i,j,x,y,m}$: マス$(i,j)$からマス$(x,y)$まで線をたどって
  距離$m$以下で到達可能.
\item $c'_{i,j,x,y,m,a}$: $c_{i,j,x,y,m}\land l_{x,y,a}$を表す補助変数.
\item $n_{i,j,a,m}$: マス$(i,j)$の方向$a$に$m$個以上黒マスがある
  (マス$(i,j)$を含まない).
\end{itemize}

\subsection{ルール}
\subsubsection{基本ルール}
\begin{itemize}
\item 数字マスは黒マスでない.
  $\N{d_{i,j}}\lor\N{b_{i,j}}$.

%\item 数字マスでも黒マスでもなければ線が通る.
%  $d_{i,j}\lor b_{i,j}
%  \lor l_{i,j,\←}
%  \lor l_{i,j,\→}
%  \lor l_{i,j,\↑}
%  \lor l_{i,j,\↓}
%  $.

\item 黒マスは線が通らない. 
  $\N{b_{i,j}}\lor\N{l_{i,j,a}}$.

\item 数字マスは線が通らない.
  $\N{d_{i,j}}\lor\N{l_{i,j,a}}$.

\item 黒マスはタテヨコに連続しない.
  \quad
  $
    (\N{b_{i,j}}\lor\N{b_{i+1,j}})
   \ \land\
    (\N{b_{i,j}}\lor\N{b_{i,j+1}}).
  $

\item マス$(i,j)$が数字マスなら $d_{i,j}$.
\item マス$(i,j)$が数字マスでなければ $\N{d_{i,j}}$.
\end{itemize}

\subsubsection{線}

\begin{itemize}
\item 線は対称.
  $\begin{alignedat}[t]{4}
   & (\N{l_{i,j,\→}} &&\lor   l_{i+1,j,\←})  &&\land
     (   l_{i,j,\→}  &&\lor\N{l_{i+1,j,\←}}) \\ {}\land{}
   & (\N{l_{i,j,\↓}} &&\lor   l_{i,j+1,\↑})  &&\land
     (   l_{i,j,\↓}  &&\lor\N{l_{i,j+1,\↑}}).
  \end{alignedat}$

\item 盤面の端とはつながらない.
  $\begin{alignedat}[t]{3}
   \text{(左右)}\quad &
   \N{l_{1,j,\←}} &&\land \N{l_{W,j,\→}} \\
   \text{(上下)}\quad &
   \N{l_{i,1,\↑}} &&\land \N{l_{i,H,\↓}}
   \end{alignedat}$

\item どのマスもたかだか2マスと線でつながる.
  $\begin{alignedat}[t]{3}
   & (\N{l_{i,j,\←}}&&\lor\N{l_{i,j,\→}}&&\lor\N{l_{i,j,\↑}}) \\ {}\land{}
   & (\N{l_{i,j,\←}}&&\lor\N{l_{i,j,\→}}&&\lor\N{l_{i,j,\↓}}) \\ {}\land{}
   & (\N{l_{i,j,\←}}&&\lor\N{l_{i,j,\↑}}&&\lor\N{l_{i,j,\↓}}) \\ {}\land{}
   & (\N{l_{i,j,\→}}&&\lor\N{l_{i,j,\↑}}&&\lor\N{l_{i,j,\↓}}).
  \end{alignedat}$

\item どのマスも1マスとだけ線でつながることはない.
  $\begin{alignedat}[t]{1}
   &(l_{i,j,\←}\lor l_{i,j,\→}\lor l_{i,j,\↑}\lor\N{l_{i,j,\↓}})\\{}\land{}
   &(l_{i,j,\←}\lor l_{i,j,\→}\lor\N{l_{i,j,\↑}}\lor l_{i,j,\↓})\\{}\land{}
   &(l_{i,j,\←}\lor\N{l_{i,j,\→}}\lor l_{i,j,\↑}\lor l_{i,j,\↓})\\{}\land{}
   &(\N{l_{i,j,\←}}\lor l_{i,j,\→}\lor l_{i,j,\↑}\lor l_{i,j,\↓}).
  \end{alignedat}$
\end{itemize}

\subsubsection{全体で一つの輪っかになる}

\begin{itemize}
\item 自分自身のみ距離0で到達可能.\quad
  $c_{i,j,i,j,0}\land \bigwedge_{(x,y)\ne(i,j)} \N{c_{i,j,x,y,0}}$.

\item $c'_{i,j,x,y,m,a}$は$c_{i,j,x,y,m}\land l_{x,y,a}$と等価.
  \quad
  $
%  &(\N{c_{i,j,x,y,m}}\lor\N{l_{x,y,a}}\lor{c'_{i,j,x,y,m,a}})\\{}\land{}
   (\N{c'_{i,j,x,y,m,a}}\lor{c_{i,j,x,y,m}})\ {}\land{}\
   (\N{c'_{i,j,x,y,m,a}}\lor{l_{x,y,a}}).
   $

\item 距離$m$以内で到達可能か距離$m$以内で到達可能なマスと線でつながっている
  ときのみ距離$m+1$以内で到達可能.
   \par\qquad
     $\N{c_{i,j,x,y,m+1}}\lor
      c_{i,j,x,y,m}\lor
      c'_{i,j,\→(x,y),m,\←}\lor
      c'_{i,j,\←(x,y),m,\→}\lor
      c'_{i,j,\↓(x,y),m,\↑}\lor
      c'_{i,j,\↑(x,y),m,\↓}$.

\item 数字マスでも黒マスでもなければ到達可能
  (最も遠いマスまでの距離は閉路の長さの半分).
  \begin{equation}
   d_{i,j}\lor b_{i,j}\lor
   d_{x,y}\lor b_{x,y}\lor
   c_{i,j,x,y,\lfloor WH/2\rfloor}.
   \label{eq:reachability}
  \end{equation}
\end{itemize}

\subsubsection{数字は矢印の方向に入る黒マスの数を表す}

\begin{itemize}
\item マス$(1,j)$より左に黒マスはない.\quad
$n_{1,j,\←,0}
\land \N{n_{1,j,\←,1}} \land\ldots\land \N{n_{1,j,\←,W}}$.

\item マス$(i+1,j)$より左の黒マスの個数.
  $\begin{aligned}[t]
   &(\N{n_{i,j,\←,m}}\lor n_{i+1,j,\←,m})\\{}\land{}
   &(\N{n_{i,j,\←,m-1}}\lor\N{b_{i,j}}\lor n_{i+1,j,\←,m})\\{}\land{}
   &(\N{n_{i+1,j,\←,m}}\lor n_{i,j,\←,m}\lor n_{i,j,\←,m-1})\\
   {}\land{}
   &(\N{n_{i+1,j,\←,m}}\lor n_{i,j,\←,m}\lor b_{i,j}).
   \end{aligned}$

\item ほかの方向も同様.

\item マス$(i,j)$に矢印$a$と数字$k$が書かれている.
  $n_{i,j,a,k}\land\N{n_{i,j,a,k+1}}$.
\end{itemize}

\section{節の削減}

%命題変数$c_{i,j,x,y,m}$,$c'_{i,j,x,y,m,a}$,及び
%これらに関連する節の個数は$O((WH)^3)$であり,
%このSAT符号化で最もコストが大きい
%(多くのヤジリン問題は$WH$が数百程度なので,
%$O((WH)^2)$ならあまり支障ないが
%$O((WH)^3)$だと少し大きい).
%これらの変数及び節の個数を減らす方法を考える.

(以下では数字マスでも黒マスでもないマスを\textbf{空白マス}と呼ぶ.)

\newcommand{\id}{\mathit{id}}
\advance\parskip by2pt
\parindent=1em

\subsection{起点マス}

上記のSAT符号化では,すべての空白マス間について
線をたどって到達できるかどうか調べているが,実際は,
適当に選んだ1つのマス(起点マス)から
他のすべての空白マスに
到達できるか調べれば十分である.
つまり,変数$c_{i,j,x,y,m}$, $c'_{i,j,x,y,m,a}$の$i,j$を
ある1つの座標値に固定してよい.
このようにすれば,変数の個数も節の個数も$O((WH)^2)$に減少する.

ただし,起点マスとして,黒マスにならないことが確実なマス
(例えば$0$が書かれた数字マスから矢印で指されているマスや,
黒マスになることが確実なマスに隣接するマス)
を選ばなければならない.
与えられたヤジリン問題に対してまずそのようなマスを1個探し,
そのマスを起点マスとして,節を生成する
(そのようなマスが見つからなければこの最適化法は適用できないが,
多くのヤジリン問題は$0$が書かれた数字マスを含むので,
適用できる場合が多いと考えられる).


\subsection{マスの順序}

上記の「起点マス」法が適用できないときに適用する.

適当な方法でマスに$1$以上$WH$以下の通し番号を付ける.
マス$(i,j)$の通し番号を$\id(i,j)$とする.
%通し番号の大小に従ってマスの順序を定義する.すなわち
%$(i,j)<(x,y) :\iff \id(i,j)<\id(x,y)$.
%
命題変数$c_{i,j,x,y,m}$(及び$c'_{i,j,x,y,m,a}$)を,
$\id(i,j)\le\id(x,y)$である$(i,j)$, $(x,y)$に対してのみ定義する.
これにより,これらの変数の個数と関連する節の個数を約半分に減らせる.

ただし,この変更により,変数$c_{i,j,x,y,m}$の意味は
「マス$(i,j)$からマス$(x,y)$まで,通し番号が$\id(i,j)$以上のマスのみ通って,
距離$m$以内で到達可能」となる.
そのため,実際は$(i,j)$から$(x,y)$まで線がつながっていても,
$c_{i,j,x,y,\lfloor WH/2\rfloor}$が
真になるとは限らない.
そこで,新しく命題変数$p_k$を導入し,
節(\ref{eq:reachability})を以下のように変更する.
%
\begin{itemize}
\item $p_k$: 通し番号が$k$以下の空白マスが存在する.
\item 通し番号が最小の空白マスから他の任意の空白マスに到達可能.
  \par\qquad
  $\N{p_{\id(i,j)}}\lor{p_{\id(i,j)-1}}\lor
   d_{x,y}\lor b_{x,y}\lor
   c_{i,j,x,y,\lfloor WH/2\rfloor}$.

  %{\small (つまり,「通し番号が最小の空白マス」が起点マスに当たり,
  %起点マスから他の空白マスに到達可能かどうかのみ調べている.)
  %どのマスが起点マスになるか,
  %SAT符号化時にはわからないので,すべての$(i,j)$に対する節を生成しておく必要がある.)
  %}
\end{itemize}
%
$p_k$の真偽は以下の節で定義する.
\begin{itemize}
\item $0$番以下の空白マスは存在しない.
  $\N{p_0}$.
\item $k-1$番以下の空白マスが存在するか$k$番のマスが空白マスであるとき,
  かつそのときのみ,$k$番以下の空白マスが存在する.
  $\begin{aligned}[t]
   &(\N{p_{\id(i,j)-1}}\lor p_{\id(i,j)}) \\ {}\land{}
   &(b_{i,j}\lor d_{i,j}\lor p_{\id(i,j)}) \\ {}\land{}
   &(\N{p_{\id(i,j)}}\lor p_{\id(i,j)-1}\lor \N{b_{i,j}}) \\ {}\land{}
   &(\N{p_{\id(i,j)}}\lor p_{\id(i,j)-1}\lor \N{d_{i,j}}).
   \end{aligned}$
\end{itemize}

\subsection{マスの偶奇}
$i+j$が偶数のときマス$(i,j)$を偶数マスと呼び,
そうでないとき奇数マスと呼ぶ.
マス$(i,j)$からマス$(x,y)$までの距離は,両マスの偶奇が一致していれば必ず偶数,
そうでなければ必ず奇数になる.
そこで,命題変数$c_{i,j,x,y,m}$(及び$c'_{i,j,x,y,m,a}$)を,
$(i,j)$と$(x,y)$の偶奇が一致していれば偶数の$m$に対してのみ,
そうでなければ奇数の$m$に対してのみ,定義する.
これにより,これらの変数の個数と関連する節の個数を約半分に減らせる.

この最適化法は「起点マス」法とも「マスの順序」法とも組み合わせて適用できる.

\end{document}
