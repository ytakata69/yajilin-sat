\documentclass[a4j]{jarticle}
\usepackage{amsmath}

\parindent=0pt
\setcounter{secnumdepth}{2}

\newcommand{\N}[1]{\overline{#1\vphantom{l}}}
\newcommand{\←}{{\leftarrow}}
\newcommand{\→}{{\rightarrow}}
\newcommand{\↑}{{\uparrow}}
\newcommand{\↓}{{\downarrow}}

\begin{document}

\section*{ヤジリンをSATソルバで解く方法}
\begin{flushright}
2018.8.24 y-takata
\end{flushright}

ヤジリンの遊び方,ルール,解き方\\
https://www.nikoli.co.jp/ja/puzzles/yajilin/

\section{準備}
盤面の幅を$W$,高さを$H$とする.
第$i$列第$j$行のマスをマス$(i,j)$と呼ぶ($1\le i\le W$, $1\le j\le H$).

$X=\{\←,\→,\↑,\↓\}$とする.
$X$の各要素はマスからマスへの関数.例えば$\→(i,j)=(i+1,j)$.
%$X(i,j)=\{(i-1,j),(i,j-1),(i+1,j),(i,j+1)\}$.

\section{SAT符号化}
\subsection{命題変数}
\begin{itemize}
\item $b_{i,j}$: マス$(i,j)$は黒.
\item $d_{i,j}$: マス$(i,j)$は数字のマス.
\item $l_{i,j,a}$: マス$(i,j)$がマス$a(i,j)$と線でつながっている($a\in X$).
\item $c_{i,j,x,y,m}$: マス$(i,j)$からマス$(x,y)$まで線をたどって
  距離$m$以下で到達可能.
\item $c'_{i,j,x,y,m,a}$: $c_{i,j,x,y,m}\land l_{x,y,a}$を表す補助変数.
\item $n_{i,j,a,m}$: マス$(i,j)$の方向$a$に$m$個以上黒マスがある
  (マス$(i,j)$を含まない).
\end{itemize}

\subsection{ルール}
\subsubsection{基本ルール}
\begin{itemize}
\item 数字マスは黒マスでない.
  $\N{d_{i,j}}\lor\N{b_{i,j}}$.

%\item 数字マスでも黒マスでもなければ線が通る.
%  $d_{i,j}\lor b_{i,j}
%  \lor l_{i,j,\←}
%  \lor l_{i,j,\→}
%  \lor l_{i,j,\↑}
%  \lor l_{i,j,\↓}
%  $.

\item 黒マスは線が通らない. 
  $\N{b_{i,j}}\lor\N{l_{i,j,a}}$.

\item 数字マスは線が通らない.
  $\N{d_{i,j}}\lor\N{l_{i,j,a}}$.

\item 黒マスはタテヨコに連続しない.
  $\begin{aligned}[t]
   & (\N{b_{i,j}}\lor\N{b_{i+1,j}}) \\
   {}\land{}
   & (\N{b_{i,j}}\lor\N{b_{i,j+1}}).
  \end{aligned}$

\item マス$(i,j)$が数字マスなら $d_{i,j}$.
\item マス$(i,j)$が数字マスでなければ $\N{d_{i,j}}$.
\end{itemize}

\subsubsection{線}

\begin{itemize}
\item 線は対称.
  $\begin{alignedat}[t]{4}
   & (\N{l_{i,j,\→}} &&\lor   l_{i+1,j,\←})  &&\land
     (   l_{i,j,\→}  &&\lor\N{l_{i+1,j,\←}}) \\ {}\land{}
   & (\N{l_{i,j,\↓}} &&\lor   l_{i,j+1,\↑})  &&\land
     (   l_{i,j,\↓}  &&\lor\N{l_{i,j+1,\↑}}).
  \end{alignedat}$

\item 盤面の端とはつながらない.
  $\begin{alignedat}[t]{3}
   \text{(左右)}\quad &
   \N{l_{1,j,\←}} &&\land \N{l_{W,j,\→}} \\
   \text{(上下)}\quad &
   \N{l_{i,1,\↑}} &&\land \N{l_{i,H,\↓}}
   \end{alignedat}$

\item どのマスもたかだか2マスと線でつながる.
  $\begin{alignedat}[t]{3}
   & (\N{l_{i,j,\←}}&&\lor\N{l_{i,j,\→}}&&\lor\N{l_{i,j,\↑}}) \\ {}\land{}
   & (\N{l_{i,j,\←}}&&\lor\N{l_{i,j,\→}}&&\lor\N{l_{i,j,\↓}}) \\ {}\land{}
   & (\N{l_{i,j,\←}}&&\lor\N{l_{i,j,\↑}}&&\lor\N{l_{i,j,\↓}}) \\ {}\land{}
   & (\N{l_{i,j,\→}}&&\lor\N{l_{i,j,\↑}}&&\lor\N{l_{i,j,\↓}}).
  \end{alignedat}$

\item どのマスも1マスとだけ線でつながることはない.
  $\begin{alignedat}[t]{1}
   &(l_{i,j,\←}\lor l_{i,j,\→}\lor l_{i,j,\↑}\lor\N{l_{i,j,\↓}})\\{}\land{}
   &(l_{i,j,\←}\lor l_{i,j,\→}\lor\N{l_{i,j,\↑}}\lor l_{i,j,\↓})\\{}\land{}
   &(l_{i,j,\←}\lor\N{l_{i,j,\→}}\lor l_{i,j,\↑}\lor l_{i,j,\↓})\\{}\land{}
   &(\N{l_{i,j,\←}}\lor l_{i,j,\→}\lor l_{i,j,\↑}\lor l_{i,j,\↓}).
  \end{alignedat}$
\end{itemize}

\subsubsection{全体で一つの輪っかになる}

\begin{itemize}
\item 自分自身のみ距離0で到達可能.
  $c_{i,j,i,j,0}\land \bigwedge_{(x,y)\ne(i,j)} \N{c_{i,j,x,y,0}}$.

\item $c'_{i,j,x,y,m,a}$は$c_{i,j,x,y,m}\land l_{x,y,a}$と等価.
  \quad
  $\begin{aligned}[t]
   &(\N{c_{i,j,x,y,m}}\lor\N{l_{x,y,a}}\lor{c'_{i,j,x,y,m,a}})\\{}\land{}
   &(\N{c'_{i,j,x,y,m,a}}\lor{c_{i,j,x,y,m}})\\{}\land{}
   &(\N{c'_{i,j,x,y,m,a}}\lor{l_{x,y,a}}).
   \end{aligned}$

\item 距離$m$以内に到達可能なら距離$m+1$以内で到達可能.
   $\N{c_{i,j,x,y,m}}\lor c_{i,j,x,y,m+1}$.
\item 到達可能なマスと線でつながっているとき到達可能.
   $\N{c'_{i,j,x,y,m,a}}\lor{c_{i,j,a(x,y),m+1}}$.
\item それらのときのみ到達可能.
   \par\qquad
     $\N{c_{i,j,x,y,m+1}}\lor
      c_{i,j,x,y,m}\lor
      c'_{i,j,\→(x,y),m,\←}\lor
      c'_{i,j,\←(x,y),m,\→}\lor
      c'_{i,j,\↓(x,y),m,\↑}\lor
      c'_{i,j,\↑(x,y),m,\↓}$.

\item 数字マスでも黒マスでもなければ互いに到達可能.
  $d_{i,j}\lor b_{i,j}\lor
   d_{x,y}\lor b_{x,y}\lor
   c_{i,j,x,y,WH-1}$.
\end{itemize}

\subsubsection{数字は矢印の方向に入る黒マスの数を表す}

\begin{itemize}
\item マス$(1,j)$より左に黒マスはない.
$n_{1,j,\←,0}
\land \N{n_{1,j,\←,1}} \land\ldots\land \N{n_{1,j,\←,W}}$.

\item マス$(i+1,j)$より左の黒マスの個数.
  $\begin{aligned}[t]
   &(\N{n_{i,j,\←,m}}\lor n_{i+1,j,\←,m})\\{}\land{}
   &(\N{n_{i,j,\←,m-1}}\lor\N{b_{i,j}}\lor n_{i+1,j,\←,m})\\{}\land{}
   &(\N{n_{i+1,j,\←,m}}\lor n_{i,j,\←,m}\lor n_{i,j,\←,m-1})\\
   {}\land{}
   &(\N{n_{i+1,j,\←,m}}\lor n_{i,j,\←,m}\lor b_{i,j}).
   \end{aligned}$

\item ほかの方向も同様.

\item マス$(i,j)$に矢印$a$と数字$k$が書かれている.
  $n_{i,j,a,k}\land\N{n_{i,j,a,k+1}}$.
\end{itemize}
\end{document}
